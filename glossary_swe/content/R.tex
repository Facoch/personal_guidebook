% File: R.tex
% Created: 2014-11-07
% Author: Tesser Paolo
% Email: p.tesser921@gmail.com
%
%
% Modification History
% Version	Modifier Date	Author			Change
% ====================================================================
% 0.0.1		2014-11-07		Tesser-Paolo	inserita sezione
% ====================================================================
% 0.0.2		2015-03-12		Tesser Paolo	inserita voce Requisito
% ====================================================================
%

\section{R}

\begin{itemize}

	\item \textbf{Raccomandazioni (DO-248)}: [PROG] bisogna cercare di evitare il non-determinismo indesiderabile come ad esempio la risoluzione dinamica di chiamata o l'allocazione e la deallocazione di memoria. Bisogna inoltre evitare la complessità cercando di tenere massima la coesione ed avere minimo accoppiamento sui dati e sul controllo. Avere programmi con un solo punto di ingresso e uno solo di uscita. Il non-determinismo ci può essere se so quali sono le parti che lo rendono tali. Bisogna stare attenti all'\textbf{overriding}. Di seguito sono elencante 10 regole d'ore prese dal libro \textbf{The Power of 10: Rules for developing safety-critical code}:
		\begin{enumerate}
			\item usare costrutti di controllo di flusso i più semplici possibile. Questo garantisce possibilità di più facile analisi statica e dinamica del codice, rendendolo più manutenibile. Implicazione è che bisogna cercare di evitare la ricorsione;
			\item usare le iterazione con limite superiore statico. In questo modo sappiamo sempre dire dove sarà il controllo;
			\item non usare allocazione dinamica di memoria dopo l'inizializzazione;
			\item limitare la lunghezza di ogni singolo sottoprogramma a 60 linee di codice;
			\item puntare a un uso intensivo di asserzioni che cercano di prevenire l'insorgere di situazioni anomale;
			\item puntare al massimo livello architetturale di data hiding;
			\item controllare che tutti i parametri in ingresso e uscita siano validi. Il controllo sui parametri in ingresso lo fa il chiamato, mentre per quelli di ritorno se ne occuperà il chiamante;
			\item limitare al massimo la compilazione condizionale;
			\item limitare al massimo l'uso dei puntatori e il livello di dereferenziazione che rende il programma intrattabile per l'analisi;
			\item il codice va compilato da subito usando il compilatore nel modo più restrittivo, adottando la regola \textbf{zero warnings} e meccanismi di \textbf{continuous integration}.
		\end{enumerate}

	\item \textbf{Requisito}: [IR] possono essere espliciti o impliciti. Secondo IEEE, è la condizione necessaria a un utente per risolvere un problema o raggiungere un obiettivo (visione dal lato del bisogno). \'E inoltre la condizione che deve essere soddisfatta o posseduta da un sistema per adempiere a un obbligo (visione dal lato della soluzione). \newline
	Il requisito è dunque necessario per: capire la domanda e capire la risposta;

	\item \textbf{Responsabile}: è uno dei ruoli di gestione di progetto. Esso rappresenta il progetto di tutto il team presso il fornitore e il committente. Viene quindi accentrata la responsabilità di scelta e approvazione. \newline
	Partecipa interamente al ciclo di vita del prodotto e sa sempre (senza bisogno di domandare nulla agli altri membri) a che punto è l'avanzamento delle attività per il raggiungimento degli obiettivi fissati. \newline
	Ha quindi responsabilità sulla pianificazione, sulla gestione delle risorse umane tramite coordinamento e relazioni esterne con il committente. \newline
 	pianificazione consiste nella definizione delle attività che il team dovrà svolgere e l'assegnamento dei tempi di esecuzione per quelle attività riuscendo a determinarne i costi di attuazione, da fornire poi come preventivo al proponente e come consuntivo nella fase finale. \newline
	Ci sono diversi strumenti per lo svolgimento di questo ruolo, come l'utilizzo di Diagrammi di Gantt e di PERT .

	\item \textbf{Ricetta generale} [PR] organizzare il sistema in componenti di complessità trattabile secondo la logica ``divide-et-impera''. Questo consente di diminuire la difficoltà di comprensione e di realizzazione, permettendo un maggiore lavoro individuale. Per facilitarsi in questo compito è utile individuare schemi architetturali e componenti riusabili. Andare poi a riconoscere le componenti terminali sotto la quale non bisogna più andare a scomporre. Infine cercare un buon bilanciamento in quanto più semplici e divise sono le componenti più complessa sarà la loro interazione e maggiore sarà l'accoppiamento;

	\item \textbf{Riuso}: [PR] utilizzare componenti già creati può permettere di capitalizzare sottoinsieme già esistenti. Impiegandoli più volte si ha un minor costo realizzativo e un minor costo di verifica. Questa progettazione però è difficile in quanto bisogna anticipare bisogni futuri. Nel breve periodo è quindi sicuramente un costo, ma farà risparmiare nel medio termine;

	\item \textbf{Ruolo}: è una funzione aziendale assegnata a progetto. E' un servizio che viene svolto per la comunità. Ne esistono di diversi e non tutti ammettono parallelismo. \newline
	Ci sono diversi ruoli che si possono raggruppare più macroscopicamente in quattro gruppi:
	\begin{enumerate}
		\item \textbf{Sviluppo}: ha la responsabilità tecnica e realizzativa;
		\item \textbf{Direzione}: ha la responsabilità decisionale (molto importante);
		\item \textbf{Amministrazione}: ha la responsabilità della gestione dei processi;
		\item \textbf{Qualità}: ha la responsabilità di gestione della qualità.
	\end{enumerate}

\end{itemize}
