% File: W.tex
% Created: 2014-11-07
% Author: Tesser Paolo
% Email: p.tesser921@gmail.com
%
%
% Modification History
% Version	Modifier Date	Author			Change
% ====================================================================
% 0.0.1		2014-11-07		Tesser-Paolo	inserita sezione
% ====================================================================
%
%

\section{W}

\begin{itemize}

	\item \textbf{Walkthrough}: [VV][analisi statica] il suo obiettivo è quello di rilevare la presenza di difetti eseguendo una lettura critica e a largo spettro di ciò che si sta esaminando senza fare nessuna assunzione. Una volta che si è eseguita questa attività e si sono rilevati gli errori più ricorrenti, bisogna prenderne nota sulla lista di controllo che servirà per le attività di analisi tramite inspection;

	\item \textbf{Work Breakdown Structure}: Breakdown significa esplosione, decomposizione. Questa attività serve per identificare tutti i compiti che dovranno essere svolti dai membri di un progetto. I compiti sono appunto delle attività che si compongono di altre sotto attività. Sono unicamente identificate e possono non essere svolte in ordine sequenziale. \newline
Si parte quindi da un macro obiettivo fino a decomporlo in compiti sempre più piccoli (ma non troppo) tali da poter essere assegnati a una singola persona e che essa possa comprenderli facilmente. \newline
Bisogna quindi stare molto attenti sia a non sottostimare il problema, sia a non sovra-stimarlo.

\end{itemize}