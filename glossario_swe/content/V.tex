% File: V.tex
% Created: 2014-11-07
% Author: Tesser Paolo
% Email: p.tesser921@gmail.com
%
%
% Modification History
% Version	Modifier Date	Author			Change
% ====================================================================
% 0.0.1		2014-11-07		Tesser-Paolo	inserita sezione
% ====================================================================
% 0.0.2		2015-02-12		Tesser Paolo	inserito vocabola: Verifica, Validazione
% ====================================================================
%

\section{V}

\begin{itemize}
	\item \textbf{Validazione}: [IR] serve a dire che tutti i requisiti sono stati soddisfatti grazie a una quantificazione. Consiste quindi nel accertare che il prodotto realizzato corrisponda alle attese. Essa quindi, a differenza delle verifica, è rivolta ai prodotti finali;

	\item \textbf{Valutazione quantitativa}: [DOC] non bisogna misurare tutto, basta quello che serve a specifiche necessità di miglioramento che vengono fissate dalla direzione in maniera precisa in appositi documenti;

	\item \textbf{Verifica}: [IR] sul come si lavora. Serve per accertare che l'esecuzione delle attività di processo non abbiano introdotto errori. La verifica è rivolta ai processi e viene svolta sui prodotti per accertare il rispetto delle regole, convenzioni e procedure;

	\item \textbf{Verificatore}: è uno dei ruoli di gestione di un progetto. Necessità come per la categoria dei programmatori di molte unità. Partecipano all'intero ciclo di vita in quanto servono a mantenere la qualità e aiutare a rimuovere gli errori. Non devono andare a rifare le cose già fatte;

\end{itemize}