% File: F.tex
% Created: 2014-11-07
% Author: Tesser Paolo
% Email: p.tesser921@gmail.com
% 
%
% Modification History
% Version	Modifier Date	Author			Change
% ====================================================================
% 0.0.1		2014-11-07		Tesser-Paolo	inserita sezione
% ====================================================================
% 
%

\section{F}
	\begin{itemize}

		\item \textbf{Facilità di prova}: [VV][analisi statica] (cioè in poco tempo faccio tanto) Esistono due strategie:
			\begin{itemize}
				\item \textbf{Investigativa}: informale e senza obiettivi prefissati di copertura;
				\item \textbf{Formale}: regolata da norme e grado di copertura fissato.
			\end{itemize}
			\noindent
			Alcuni costrutti del linguaggio complicano questa analisi. In particolare possono essere d'ostacolo: la risoluzione dinamica di chiamata (dispatching), la conversione forzata tra tipi (casting) o l'eccezioni predefinite;

		\item \textbf{Forme di analisi}: [VV] esistono due principali forme di analisi per effettuare la verifica:
			\begin{itemize}
				\item \textbf{Analisi statica}: non richiede l'esecuzione di alcuna parte del prodotto SW, viene quindi maggiormente impiegata quando il sistema non è ancora disponibile;
				\item \textbf{Analisi dinamica}: richiede l'esecuzione del programma. Viene effettuata tramite prove e test. Questa viene usata sia nelle attività di verifica sia di validazione.
			\end{itemize}

		\item \textbf{Framework}: [PR] insieme integrato di componenti SW prefabbricate. Prima della OO erano chiamate librerie. Sono bottom-up perché fatti da codice già sviluppato, ma anche top-down perché impongono uno stile architetturale. Sono comodi come base di diverse applicazioni entro un dato dominio;

	\end{itemize}