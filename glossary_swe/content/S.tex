% File: S.tex
% Created: 2014-11-07
% Author: Tesser Paolo
% Email: p.tesser921@gmail.com
% 
%
% Modification History
% Version	Modifier Date	Author			Change
% ====================================================================
% 0.0.1		2014-11-07		Tesser Paolo	inserita sezione
% ====================================================================
% 0.0.2		2015-02-03		Tesser Paolo	agginte voci: Servizio
% ====================================================================
% 0.0.3		2015-02-10		Tesser Paolo	agginte voci: Sistema di qualità, Standard
% ====================================================================
% 0.0.4		2015-02-24		Tesser Paolo	aggiunta voce: Sistema gestione qualità, strumenti di valutazione
% ====================================================================
% 0.0.5		2015-02-25		Tesser Paolo	aggiunta voci: SPY e SPICE
% ====================================================================
%

\section{S}

\begin{itemize}
	\item \textbf{Schemi (pattern) architetturali}: [PR] sono soluzioni fattorizzate per problemi ricorrenti. Questo approccio è tipico dell'ingegneria classica. La soluzione deve riflettere il contesto e deve essere credibile;

	\item \textbf{Scrum}: TO DO;

	\item \textbf{SEMAT}: Software Engineering Method and Theory è una comunità industriale con lo scopo di migliorare le pratiche dell'ingegneria del software, ridefinendola come una disciplina rigorosa;

	\item \textbf{Servizio}: è un mezzo per fornire valore all'utente, agevolando il raggiungimento dei suoi obiettivi senza doverne sostenere gli specifici costi e rischi;

	\item \textbf{Sistema di qualità}: [QS] è formato dalle strutture organizzative, dalle responsabilità, dalle procedure e i procedimenti messi in atto per raggiungere la qualità (secondo ISO 9000:2005). La gestione aziendale deve quindi pianificare e controllare gli obiettivi che si vuole raggiungere in modo tale da poter applicare un miglioramento continuo.
		\begin{itemize}
			\item \textbf{Pianificare la qualità}: attraverso attività mirate a fissare gli obiettivi di qualità, i processi e le risorse necessarie per conseguirli (secondo ISO 9000). Possono essere eseguite trasversalmente all'intera organizzazione, se si vuole pianificare la qualità a livello aziendale. Altrimenti possono essere eseguite verticalmente al singolo progetto;
			\item \textbf{Controllare la qualità}: attraverso le attività pianificate e attuate affinché il prodotto soddisfi i requisiti attesi (secondo ISO 9000);
		\end{itemize}

	\item \textbf{Sistema di gestione qualità}: [QP] viene stabilito dalla direzione dell'azienda ed essere trasversale ad ogni processo. Il suo compito è quello di gestire la qualità e riferire ai dirigenti. Ha il compito di fornire i seguenti documenti:
		\begin{itemize}
			\item \textbf{Politica per la qualità}: dove vengono illustrati i motivi per i quali perseguiamo il raggiungimento della qualità;
			\item \textbf{Manuale della qualità}: dove viene definito il sistema di gestione della qualità di un'organizzazione. \'E dunque una visione ad alto livello (secondo ISO 9000);
			\item \textbf{Piano della qualità (di qualifica)}: dove vengono definiti gli elementi del SGQ e le risorse che devono essere applicate in uno specifico caso (secondo ISO 9000). Concretizza dunque il manuale precedente a livello di progetto, quindi con vincoli di risorse e di tempo.
		\end{itemize}

	\item \textbf{Sistematico}: costante, che fa sempre la stessa cosa in determinate situazioni, applicando un comportamento, che deve tendere ad esser best, ripetitivo;
	\item \textbf{SPICE e ISO 15504}: [QP] è il corrispettivo europeo del CMMI, solo che più strutturato e filosofico. Definisce un modello per la valutazione del livello di ``maturità'' dei processi di una organizzazione del settore IT. \newline
	La norma definisce un modello su due dimensioni: quella sul processo, in termini di obiettivi e di risultati misurabili attesi, e quella sul profilo di capacità dei processi, caratterizzata da una serie di attributi che ne distinguono il livello di maturità rispetto a una scala di valori [0-5]. ;


	\item \textbf{Specifica Tecnica}: [DOC] documento che descrive l'architettura logica del prodotto. Viene redatta dopo avere steso l'Analisi dei Requisiti. Mostra ciò che il sistema deve fare, non andando a fissare i dettagli implementativi. Inizialmente si ha un approccio top-down. Per ogni componente va specificato:
		\begin{itemize}
			\item la funzione svolta
			\item tipo di dati in ingresso
			\item tipo di dati in uscita
			\item risorse logiche e fisiche necessarie al suo funzionamento
		\end{itemize}
		\noindent
		La definizione delle componenti e la loro decomposizione può avvenire in due modi diversi:
		\begin{itemize}
			\item \textbf{Decomposizione funzionale}: si inizia a redarla in maniera top-down; TODO
			\item \textbf{Decomposizione ad oggetti}: si inizia a redarla in continuità logica e di notazione con l'analisi dei requisiti in stile OO. TODO
		\end{itemize}


	\item \textbf{SPY}: [QP] TO DO;

	\item \textbf{Stack}: [VV] è l'aerea di memoria che i sottoprogrammi usano per immagazzinare dati locali, temporanei e indirizzi di ritorno generati dal compilatore. Ogni flusso di controllo ha un suo proprio stack. La sua grandezza cresce con l'annidamento di chiamate di procedura;

	\item \textbf{Stakeholder (portatore di interesse)}: è l'insieme di persone coinvolte nel ciclo di vita del SW con influenza sul prodotto. \newline
	Possono essere chi usa lo usa, chi lo paga o chi lo sviluppa;
	\item \textbf{Standard}: [QS] sono una raccolta di best practice per evitare le ripetizioni di errori passati. Essi sono la chiave del miglioramento continuo, effettuato tramite l'attuazione del PDCA. \newline
	Forniscono un elemento di continuità, infatti i nuovi assunti possono comprendere più facilmente l'organizzazione aziendale a partire dagli standard di qualità in uso. \newline
	Il personale però gli può percepire come irrilevanti o bloccanti, e se la loro attuazione avvenisse in maniera rigida e chiusa potrebbe comportare ad un eccesso di burocrazia. Proprio per questo bisogna cercare di dare norme snelle e comprensibili, il più delle volte automatizzate per non incombere in attività manuali ripetitive;

	\item \textbf{Stati di progresso per SEMAT}: [IS] Il SEMAT definisce alcuni livelli per vedere a che stato sono i requisiti trovati. Verranno elencati dal meno completo al livello di soddisfacimento massimo.
		\begin{enumerate}
			\item \textbf{Conceived}: il committente è identificato e gli stakeholder vedono sufficiente opportunità per il progetto dopo aver effettuato lo studio di fattibilità. In questo momento non si ha ancora una Requirements Baseline;
			\item \textbf{Bounded}: il problema è stato delimitato e i macro bisogni sono chiari. Sono fissati i meccanismi di gestione dei requisiti. So quindi cosa deve stare all'interno e cosa all'esterno della Baseline;
			\item \textbf{Coherent}; non ci sono più ambiguità. I requisiti sono classificati e quelli obbligatori sono chiari e ben definiti. Siamo al punto di avere quasi una baseline;
			\item \textbf{Acceptable}: i requisiti fissati definiscono un sistema soddisfacente per gli stakeholder;
			\item \textbf{Addressed}: il prodotto soddisfa i principale requisiti al punto da poter meritare rilascio e uso;
			\item \textbf{Fulfilled}: il prodotto soddisfa abbastanza i principali requisiti da meritare la piena approvazione degli stakeholder. Posso dirlo però solo quando ho in mano gli elementi della validazione.
		\end{enumerate}

	\item \textbf{Stati di progresso per SEMAT}: [PR]
		\begin{itemize}
			\item \textbf{Architecture selected}: viene selezionata un'architettura tecnicamente adatta al problema e si sono selezionate le tecnologie necessarie;
			\item \textbf{Demonstrable}: dimostrazione delle principali caratteristiche dell'architettura con l'approvazione degli stakeholder. Inoltre sono state prese le decisioni sulle principali interfacce e configurazioni di sistema;
			\item \textbf{Usable}: il sistema è utilizzabile e ha le caratteristiche desiderate. Può essere quindi operato dagli utenti. Le funzionalità richieste sono state verificate e validate;
			\item \textbf{Ready}: la documentazione per l'utente è pronta e gli stakeholder hanno accettato il prodotto e vogliono che diventi operativo.


	\item \textbf{Strategie di integrazione}: [VV][analisi dinamica] le attività devono essere pianificate all'indietro per lasciare spazio alle verifiche. Nel repository entreranno solo cose verificate. Ad esempio possono avere degli \textbf{hooks} che mi avviano i test prima che possa fare una commit o un push, e se falliscono non mi permettono di portare a termine le operazioni. Alcune strategie di integrazione sono:
		\begin{itemize}
			\item \textbf{assemblare parti in modo incrementale}: in questo modo i difetti rilevati in un test sono più probabilmente da attribuirsi alla parte ultima aggiunta;
			\item \textbf{assemblare produttori prima dei consumatori}
			\item \textbf{assemblare in modo che ogni passo di integrazione sia reversibile}: in questo modo è possibile retrocedere verso uno stato noto e sicuro.
		\end{itemize}

		\end{itemize}
	\item \textbf{Strumenti di valutazione}: [QP] di seguito vengono elencati alcuni strumenti e riferimenti per la valutazione e il miglioramento dei processi. Verranno poi descritti in apposite voci riportanti lo stesso nome presente nella suddetta lista.
		\begin{itemize}
			\item \textbf{SW Process Assessment \& Improvement (SPY)}: serve per dare una valutazione oggettiva dei processi di una organizzazione. Fornisce quindi un giudizio di maturità e individua le azione migliorative da attuare;
			\item \textbf{CMM (CMMI)}: Capability Maturity Model poi evoluto in CMMI con la I di Integration;
			\item \textbf{SPICE e ISO/UEC 15504}: Software Process Improvement Capability dEtermination, nato per armonizzare SPY con ISO/IEC 12207 e ISO 9001, confluito infine in ISO/IEC 15504.
		\end{itemize}
	\item \textbf{Studio di fattibilità}: [IS] è un'attività che deve essere svolta internamente al gruppo. Serve per valutare i rischi, i costi e i benefici dello svolgimento di un progetto, proposto da un cliente o ha necessità interna, con le conoscenza immediatamente disponibili, senza dover effettuare quindi ricerche impegnative. Ha il compito quindi di decidere se procedere o meno nelle attività con l'obiettivo di restare entro un costo massimo prefissato. \newline
	Questo studio deve valutare diversi aspetti:
		\begin{itemize}
			\item \textbf{Fattibilità tecnico-organizzativa}: ossia quali strumenti servono per la realizzazione, quali sono le soluzioni algoritmiche, architetturali e quali sono le piattaforme idonee per l'esecuzione;
			\item \textbf{Rapporto costi/benefici}: bisogna confrontare il il mercato attuale con quello futuro e valutare il costo di produzione rispetto alla redditività dell'investimento;
			\item \textbf{Individuazione dei rischi}: bisogna analizzare la complessità e le incertezze del progetto;
			\item \textbf{Valutazione delle scadenze temporali}: risorse disponibili rispetto a quelle necessarie.
		\end{itemize}
\end{itemize}