% File: D.tex
% Created: 2014-11-07
% Author: Tesser Paolo
% Email: p.tesser921@gmail.com
% 
%
% Modification History
% Version	Modifier Date	Author			Change
% ====================================================================
% 0.0.1		2014-11-07		Tesser Paolo	inserita sezione
% ====================================================================
% 0.0.2		2015-02-03		Tesser Paolo	inserita parola Documentazione
% ====================================================================
% 0.0.3		2015-02-10		Tesser Paolo	iniziata stesura vocaboli. Design Pattern
% ====================================================================
%

\section{D}

\begin{itemize}
	\item \textbf{Design Pattern}: [PR] riportano soluzioni a problemi che sono già state sviluppate e che si sono evolute nel corso del tempo in modo tale che il progettista non debba riscoprire ogni volta cose già esistenti. La forma con cui si presentano è succinta e facilmente applicabile. Ci vuole un notevole sforzo ad applicarli, ma ciò viene ripagato dall'incremento nella flessibilità e nella riusabilità della soluzione adottata. \newline
	I pattern non riguardano solo il mondo dei software, ma svariati campi, come ad esempio quelli per la stesura di novelle romantiche o fiabe. \newline
	Per quanto riguarda il mondo software vanno inseriti nella Specifica Tecnica, nella Definizione di Prodotto e nel Piano di Qualifica (?). \newline
	Generalmente un pattern ha quattro elementi essenziali:
		\begin{enumerate}
			\item \textbf{Nome del pattern}: nome simbolico che ci aiuta a inquadrare il problema, permettendoci di ampliare il vocabolario di progetto;
			\item \textbf{Problema}: descrive la situazione al quale applicare il pattern;
			\item \textbf{Soluzione}: descrive gli elementi che costituiscono il progetto, le loro relazioni, responsabilità e collaborazioni;
			\item \textbf{Conseguenze}: sono i risultati e i vincoli che si ottengono applicando il pattern.
		\end{enumerate}
		\noindent
	La classificazione di essi avviene a seconda dello scopo e del loro raggio d'azione.
		\begin{itemize}
			\item \textbf{Scopo}: ciò che il pattern fa. Possono essere di tipo:
				\begin{itemize}
					\item \textbf{Creazionale}: che riguarda il processo di creazione di oggetti;
					\item \textbf{Strutturale}: che riguarda la composizione di classi e oggetti;
					\item \textbf{Comportamentale}: che riguarda il modo in cui le classi o oggetti interagiscono reciprocamente e distribuiscono fra loro le responsabilità;
					\item \textbf{Architetturale}: che riguarda l'architettura globale tra i diversi componenti. Guida la scomposizione in diversi sottoinsiemi. Questi sottoinsiemi sono formati dai design pattern del tipo citato precedentemente.
				\end{itemize}
			\item \textbf{Raggio d'azione}: specifica se il pattern si applica a classi o a oggetti.
		\end{itemize}
		\noindent
	I design pattern ci permettono appunto di risolvere alcuni problemi. Eccone alcuni di essi:
		\begin{itemize}
			\item \textbf{Trovare gli oggetti appropriati}: TO DO;
			\item \textbf{Determinare la granularità degli oggetti}: TO DO;
			\item \textbf{Definire le interfacce degli oggetti}: TO DO;
			\item \textbf{Definire le implementazioni degli oggetti}: TO DO;
			\item \textbf{Mettere in pratica il riuso}: TO DO;
			\item \textbf{Delega}: TO DO;
			\item \textbf{Strutture correlate in compilazione e durante l'esecuzione}: TO DO;
			\item \textbf{Progettare per il cambiamento}: TO DO;
			\item \textbf{Delega}: TO DO;
		\end{itemize}



	\item \textbf{Determinismo}: è la concezione della realtà secondo la quale tutti i fenomeni del mondo sono collegati l’un l’altro e si verificano secondo un ordine necessario e invariabile (il che esclude la presenza del libero arbitrio). \newline
In informatica possiamo applicare questo concetto al fatto che una funzione, o un determinato evento, dia sempre un certo output dato sempre lo stesso input e che quel risultato sarà disponibile in un tempo quantificabile;

	\item \textbf {Diagramma di GANTT}: [PM] è uno degli strumenti di supporto alle attività di pianificazione eseguite dal responsabile di progetto. \newline
Serve per gestire la dislocazione temporale delle attività, per rappresentarne la durata, preventivata e quella poi effettiva, la sequenzialità o il parallelismo. \newline
L'asse delle ascisse è destinata allo scorrere del tempo ed è lunga quanto la durata del progetto. \newline
L'asse delle ordinate invece è riservata alle mansioni che costituiscono il progetto. \newline
Le barre orizzontali rappresentate servono ad indicare la quantità di tempo riservato ad una determinata mansione. Ne deriva quindi un calendario delle attività. \newline
Questo metodo offre quindi un'idea più chiara sull'andamento del progetto e permette di controllare meglio i budget di risorse e conseguentemente i costi;

	\item \textbf {Diagramma di PERT}: (Programme Evaluation and Review Technique) [PM] è uno degli strumenti di supporto alle attività di pianificazione eseguite dal responsabile di progetto. \newline
Serve per arricchire le informazioni date dal diagramma di Gantt. In esso vengono gestite principalmente le dipendenze tra un'attività e un'altra, andando a segnare il tempo di inizio e quello di fine. Tra le attività correlate si potrà dunque verificare se è presente dello \textbf{Slack}, cioè del tempo di margine in caso l'attività precedente non riesca ad essere eseguita nei tempi prefissati. Bisogna stare attenti a come si gestisce ciò perché molto slack causa inattività e non consente ritardo finale.   \newline
Questa tecnica, congiunta allo studio del Cammino critico, permette di visualizzare meglio i collegamenti tra le varie attività e consenti di valutare più facilmente i percorsi critici;

	\item \textbf{Disciplina}: insieme di regole e procedure che attuate conducono a un risultato conforme a determinate aspettative;
	\item \textbf{Disciplinato}: fissato un principio, lo segue sempre senza aggiungere o fare cose che non sono state regolamentate precedentemente da norme di gruppo;
	\item \textbf{Documentazione}: [AM] riguarda tutto ciò che documenta le attività di un progetto, riguardo al prodotto e al processo.
	Per essere utile deve sempre essere disponibile, corretta nei contenuti, verificata, approvata e aggiornata. La diffusione di questi documenti deve essere controllata. Ogni documento infatti deve avere una lista di distribuzione, coerente con i suoi scopi, per non generare offuscamento in chi la riceve.
	\item \textbf{Duck typing}: nei linguaggi di programmazione ad oggetti si riferisce ad uno stile di tipizzazione dinamica dove la semantica di un oggetto è determinata dall'insieme corrente dei suoi metodi e dalla sue proprietà piuttosto che dal fatto di estendere una particolare classe o implementare una specifica interfaccia. \newline
Alcuni linguaggi a cui si riesce ad applicare bene questo principio sono JavaScript o PHP.
Il nome si riferisce al duck test attribuito a James Whitcomb Riley il quale si può esporre attraverso la frase: ``Quando vedo un uccello camminare come un'anatra, nuotare come un'anatra e fare il verso come l'anatra, posso chiamare quell'uccello anatra.".

\end{itemize}
