% File: Q.tex
% Created: 2014-11-07
% Author: Tesser Paolo
% Email: p.tesser921@gmail.com
%
%
% Modification History
% Version	Modifier Date	Author			Change
% ====================================================================
% 0.0.1		2014-11-07		Tesser-Paolo	inserita sezione
% ====================================================================
% 0.0.2		2015-02-10		Tesser Paolo	inseriti vocaboli: Qualità
% ====================================================================
%

\section{Q}

\begin{itemize}
	\item \textbf{Qualità}: [QS] è l'insieme delle caratteristiche di un'entità (prodotto, processo, servizio) che ne determinano la capacità di soddisfare esigenze espresse e implicite. La qualità serve per intervenire su alcune aree per e la sua visione deve essere:
		\begin{itemize}
			\item \textbf{intrinseca}: conforme con i requisiti e idoneità all'uso (sempre presenti a prescindere);
			\item \textbf{relativa}: deve soddisfare il cliente;
			\item \textbf{quantitativa}: ricevere un livello di qualità per confronto.
		\end{itemize}
		\noindent
		Questi concetti si legano fortemente a quello di valutazione. Servono dunque mezzi oggettivi per dare un valore a ciò che si è fatto. \newline
		La ricetta per raggiungere la qualità (a livello di PDCA) consiste nel pianificare bene ciò che deve essere realizzato e come andrà controllato rispetto agli obiettivi di miglioramento. In seguito bisognerà controllare, per poter conoscere e intervenire in tempo. \newline
		A volte al fornitore è richiesto che il prodotto rispetti alcune norme, sia a livello di prodotti sia di processi in modo che il cliente sia tutelato rispetto all'uso o al valore dei prodotti. \newline
		La qualità dovrà essere quindi: esterna, interna ed in uso. Le prime due sono fortemente connesse e riguardano: le funzionalità, l'affidabilità, l'efficienza, l'usabilità, la manutenibilità e la portabilità. Quella interna sarà fonte di requisiti, anche se non espliciti, e la più vasta rispetto alla interna. Deriva infatti da scelte di progettazione codifica, verifica e che si vede solo attraverso una revisione critica. Quella interna si osserva invece tramite esecuzione del prodotto;

		\item \textbf{Qualità architetturali}: [PR][TO DO];

	\item \textbf{Quantificabile}: di cui è possibile misurarne l'efficacia e l'efficienza, anche a priori;

\end{itemize}