% File: I.tex
% Created: 2014-11-07
% Author: Tesser Paolo
% Email: p.tesser921@gmail.com
%
%
% Modification History
% Version	Modifier Date	Author			Change
% ====================================================================
% 0.0.1		2014-11-07		Tesser-Paolo	inserita sezione
% ====================================================================
% 0.0.2		2015-02-05		Tesser Paolo	inseriti vocaboli: Infrastruttura
% ====================================================================
% 0.0.3		2015-02-11		Tesser Paolo	inseriti vocaboli: Idioma
% ====================================================================
%

\section{I}

\begin{itemize}
	\item \textbf{Idioma}: [PR] è una soluzione specifica ad un linguaggio. Legato quindi alla tecnologia. Esso rinuncia alla genericità della soluzione basata su un design pattern a favore di un implementazione che sfrutta le caratteristiche e le potenzialità del linguaggio;
	\item \textbf{Infrastruttura}: è l'insieme dei servizi offerti sotto la responsabilità dell'amministratore, necessaria allo svolgimento di un progetto. Questa molte volte è trasversale, in parte o totalmente, a più progetti. \'E composta dalle risorse HW (server, rete, postazione di lavoro) e da quelle SW (ambiente di sviluppo, prova, gestione, documentazione);
	\item \textbf{Incremento}: procedere in questo modo significa aggiungere ad un impianto base. Si inserisce, ma non si toglie;

	\item \textbf{Iterazione}: procedere in questo modo significa operare raffinamenti o rivisitazioni. A differenza della tecnica incrementale, questa può essere distruttiva. \newline
Devo inoltre fissare un numero preciso di iterazioni da compiere e per farlo ha bisogno di condizioni di uscita forti dal ciclo;

\end{itemize}