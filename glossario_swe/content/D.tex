% File: D.tex
% Created: 2014-11-07
% Author: Tesser Paolo
% Email: p.tesser921@gmail.com
% 
%
% Modification History
% Version	Modifier Date	Author			Change
% ====================================================================
% 0.0.1	2014-11-07	Tesser-Paolo		inserita sezione 
% ====================================================================
% 
%

\section{D}

\begin{itemize}
	\item \textbf{Determinismo} : è la concezione della realtà secondo la quale tutti i fenomeni del mondo sono collegati l’un l’altro e si verificano secondo un ordine necessario e invariabile (il che esclude la presenza del libero arbitrio). \\
In informatica possiamo applicare questo concetto al fatto che una funzione, o un determinato evento, dia sempre un certo output dato sempre lo stesso input e che quel risultato sarà disponibile in un tempo quantificabile;

	\item \textbf {Diagramma di GANTT} : è uno degli strumenti di supporto alle attività di pianificazione eseguite dal responsabile di progetto. \\
Serve per gestire la dislocazione temporale delle attività, per rappresentarne la durata, preventivata e quella poi effetiva, la sequenzialità o il parallelismo. \\
L'asse delle ascisse è destinata allo scorrere del tempo ed è lunga quanto la durata del progetto. \\
L'asse delle ordinate invece è riservata alle mansioni che costituiscono il progetto. \\
Le barre orizzontali rappresentate servono ad indicare la quantità di tempo riservato ad una determinata mansione. Ne deriva quindi un calendario delle attività. \\
Questo metodo offre quindi un'idea più chiara sull'andamento del progetto e permette di controllare meglio i budget di risorse e conseguentemente i costi;

	\item \textbf {Diagramma di PERT} : (Programme Evaluation and Review Technique) è uno degli strumenti di supporto alle attività di pianificazione eseguite dal responsabile di progetto. \\
Serve per arricchire le informazioni date dal diagramma di Gantt. In esso vengono gestite principalmente le dipendenze tra un'attività e un'altra, andando a segnare il tempo di inizio e quello di fine. Tra le attività correlate si potrà dunque verificare se è presente dello \textbf{Slack}, cioè del tempo di margine in caso l'attività precedente non riesca ad essere eseguita nei tempi prefissati. Bisogna stare attenti a come si gestisce ciò perché molto slack causa inattività e non consente ritardo finale.   \\
Questa tecnica, congiunta allo studio del Cammino critico, permette di visualizzare meglio i collegamenti tra le varie attibità e consenti di valutare più facilemte i percorsi critici;

	\item \textbf{Disciplina} : insieme di regole e procedure che attuate conducono a un risultato conforme a determinate aspettative;
	\item \textbf{Disciplinato} : fissato un principio, lo segue sempre senza aggiungere o fare cose che non sono state regolamentate precedentemente da norme di gruppo;	
	\item \textbf{Duck typing} : nei linguaggi di programmazione ad oggetti si riferisce ad uno stile di tipizzazione dinamica dove la semantica di un oggetto è determinata dall'insieme corrente dei suoi metodi e dalla sue proprietà piuttosto che dal fatto di estendere una particolare classe o implementare una specifica interfaccia. \\
Alcuni linguaggi a cui si riesce ad applicare bene questo principio sono JavaScript o PHP.
Il nome si riferisce al duck test attribuito a James Whitcomb Riley il quale si può esporre attriverso la frase: "Quando vedo un uccello camminare come un'anatra, nuotare come un'anatra e fare il verso come l'anatra, posso chiamare quell'uccello anatra.".

\end{itemize}
