% File: S.tex
% Created: 2014-11-07
% Author: Tesser Paolo
% Email: p.tesser921@gmail.com
% 
%
% Modification History
% Version	Modifier Date	Author			Change
% ====================================================================
% 0.0.1		2014-11-07		Tesser Paolo	inserita sezione
% ====================================================================
% 0.0.2		2015-02-03		Tesser Paolo	agginte voci: Servizio
% ====================================================================
% 0.0.3		2015-02-10		Tesser Paolo	agginte voci: Sistema di qualità, Standard
% ====================================================================
% 0.0.4		2015-02-24		Tesser Paolo	aggiunta voce: Sistema gestione qualità, strumenti di valutazione
% ====================================================================
%

\section{S}

\begin{itemize}
	\item \textbf{Scrum}: TO DO;
	\item \textbf{SEMAT}: Software Engineering Method and Theory è una comunità industriale con lo scopo di migliorare le pratiche dell'ingegneria del software, ridefinendola come una disciplina rigorosa;
	\item \textbf{Servizio}: è un mezzo per fornire valore all'utente, agevolando il raggiungimento dei suoi obiettivi senza doverne sostenere gli specifici costi e rischi;
	\item \textbf{Sistema di qualità}: [QS] è formato dalle strutture organizzative, dalle responsabilità, dalle procedure e i procedimenti messi in atto per raggiungere la qualità (secondo ISO 9000:2005). La gestione aziendale deve quindi pianificare e controllare gli obiettivi che si vuole raggiungere in modo tale da poter applicare un miglioramento continuo.
		\begin{itemize}
			\item \textbf{Pianificare la qualità}: attraverso attività mirate a fissare gli obiettivi di qualità, i processi e le risorse necessarie per conseguirli (secondo ISO 9000). Possono essere eseguite trasversalmente all'intera organizzazione, se si vuole pianificare la qualità a livello aziendale. Altrimenti possono essere eseguite verticalmente al singolo progetto;
			\item \textbf{Controllare la qualità}: attraverso le attività pianificate e attuate affinché il prodotto soddisfi i requisiti attesi (secondo ISO 9000);
		\end{itemize}

	\item \textbf{Sistema di gestione qualità}: [QP] viene stabilito dalla direzione dell'azienda ed essere trasversale ad ogni processo. Il suo compito è quello di gestire la qualità e riferire ai dirigenti. Ha il compito di fornire i seguenti documenti:
		\begin{itemize}
			\item \textbf{Politica per la qualità}: dove vengono illustrati i motivi per i quali perseguiamo il raggiungimento della qualità;
			\item \textbf{Manuale della qualità}: dove viene definito il sistema di gestione della qualità di un'organizzazione. \'E dunque una visione ad alto livello (secondo ISO 9000);
			\item \textbf{Piano della qualità (di qualifica)}: dove vengono definiti gli elementi del SGQ e le risorse che devono essere applicate in uno specifico caso (secondo ISO 9000). Concretizza dunque il manuale precedente a livello di progetto, quindi con vincoli di risorse e di tempo.
		\end{itemize}

	\item \textbf{Sistematico}: costante, che fa sempre la stessa cosa in determinate situazioni, applicando un comportamento, che deve tendere ad esser best, ripetitivo;
	\item \textbf{Stakeholder (portatore di interesse)}: è l'insieme di persone coinvolte nel ciclo di vita del SW con influenza sul prodotto. \newline
	Possono essere chi usa lo usa, chi lo paga o chi lo sviluppa;
	\item \textbf{Standard}: [QS] sono una raccolta di best practice per evitare le ripetizioni di errori passati. Essi sono la chiave del miglioramento continuo, effettuato tramite l'attuazione del PDCA. \newline
	Forniscono un elemento di continuità, infatti i nuovi assunti possono comprendere più facilmente l'organizzazione aziendale a partire dagli standard di qualità in uso. \newline
	Il personale però gli può percepire come irrilevanti o bloccanti, e se la loro attuazione avvenisse in maniera rigida e chiusa potrebbe comportare ad un eccesso di burocrazia. Proprio per questo bisogna cercare di dare norme snelle e comprensibili, il più delle volte automatizzate per non incombere in attività manuali ripetitive;
	\item \textbf{Strumenti di valutazione}: [QP] di seguito vengono elencati alcuni strumenti e riferimenti per la valutazione e il miglioramento dei processi. Verranno poi descritti in apposite voci riportanti lo stesso nome presente nella suddetta lista.
		\begin{itemize}
			\item \textbf{SW Process Assessment & Improvement (SPY)}: TO DO;
			\item \textbf{CMM (Capability Maturity Model)(CMMI)}: TO DO;
			\item \textbf{SPICE e ISO/UEC 15504}: TO DO.
		\end{itemize}
\end{itemize}