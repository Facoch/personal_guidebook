% File: F.tex
% Created: 2014-11-07
% Author: Tesser Paolo
% Email: p.tesser921@gmail.com
% 
%
% Modification History
% Version	Modifier Date	Author			Change
% ====================================================================
% 0.0.1		2014-11-07		Tesser Paolo	inserita sezione
% ====================================================================
% 
%

\section{E}

\begin{itemize}
	\item \textbf{Efficacia}: è la capacità di raggiungere l'obiettivo specifico che si cercava di raggiungere applicando una determinata azione. \newline
Viene determinata dal grado di conformità del prodotto rispetto alle norme vigenti e agli obiettivi prefissati;

	\item \textbf{Efficienza}: è la capacità di produrre qualcosa, impiegando il minor numero di risorse che siano esse tempo, denaro o persone durante l'esecuzione delle attività richieste; \newline

	\item \textbf{Error}: [VV][analisi dinamica] un errore è uno stato del sistema che è discrepante con le attese corrette e può indurre a una \textbf{Failure}. Essi sono causati da un \textbf{Fault};

	\item \textbf{Elementi di prova}: [VV][analisi dinamica]
		\begin{itemize}
			\item \textbf{Caso di prova (test case)}: è una tripla <ingresso, uscita, ambiente> che rende un test ripetibile;
			\item \textbf{Batterie di prova (test suite)}: è un insieme di casi di prova;
			\item \textbf{Procedura di prova}: è l'insieme di istruzioni tali che la sequenza di input venga catturata, eseguita e produca un log in output;
			\item Prova = <Procedura, caso di prova>
			\item \textbf{Oracolo}: si trova nella specifica tecnica. \'E un decisore noto a priori che serve a generare risultati attesi.
		\end{itemize}
\end{itemize}