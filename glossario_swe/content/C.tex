% File: c.tex
% Created: 2014-11-07
% Author: Tesser Paolo
% Email: p.tesser921@gmail.com
% 
%
% Modification History
% Version	Modifier Date	Author			Change
% ====================================================================
% 0.0.1	2014-11-07	Tesser-Paolo		inserita sezione 
% ====================================================================
% 
%

\section{C}

\begin{itemize}
	\item \textbf{Casi d'uso} : tecnica usata nei processi di ingegneria del software per effettuare in maniera esaustiva e non ambigua, la raccolta dei requisiti al fine di produrre software di qualità. Consiste nel valutare ogni requisito, focalizzandosi sugli attori che interagiscono col sistema e valutandone le varie interazioni. Il documento dei casi d’uso, individua e descrive gli scenari elementari di utilizzo del sistema, da parte degli attori che si interfacciano con esso.

	\item \textbf{Ciclo di Deming} : è un modello studiato per il miglioramento continuo della qualità in un’ottica a lungo raggio. Serve per promuovere una cultura della qualità che è tesa al miglioramento continuo dei processi e all’utilizzo ottimale delle risorse. Questo strumento parte dall’assunto che per il raggiungimento del massimo della qualità sia necessaria la costante interazione tra ricerca, progettazione, test, produzione e vendita.
Per migliorare la qualità e soddisfare il cliente, le quattro fasi devono ruotare costantemente, tenendo come criterio principale la qualità.
La sequenza logica dei quattro punti ripetuti per un miglioramento continuo è la seguente:
	\begin{itemize}
		\item P: Plan. Pianificazione;
		\item D: Do. Esecuzione del programma, dapprima in contesti circoscritti;
		\item C: Check. Test e controllo, studio e raccolta dei risultati e dei riscontri;
		\item A: Act. Azione per rendere definitivo e/o migliorare il processo.
	\end{itemize}

	\item \textbf{Closure} : nei linguaggi di programmazione è una astrazione che combina una funzione con le variabili libere presenti nell'ambiente in cui è definita. \\

	\item \textbf{Controllo di versione} :  è la gestione di versioni multiple di un insieme di informazioni. I documenti che vengono gestiti attraverso questa tecnica sono tutti quelli che hanno o necessitano di avere una storia.\\
Questi strumenti (Git, SVN, Mercurial) ci permettono di muoverci attraverso le varie modifiche che sono state effettuate su file digitali come del codice sorgente, disegni tecnici, documentazione o altro, e, in caso di necessità, ripristinare quelli antecedenti alla versione attuale. \\
Tutto ciò serve per correggere errori o ritornare a stati in cui non ce ne erano. Ci permette inoltre di procedere su rami diversi in modo da non entrare in conflitto con sezioni stabili dello sviluppo;

\end{itemize}