% File: A.tex
% Created: 2014-11-07
% Author: Tesser Paolo
% Email: p.tesser921@gmail.com
% 
%
% Modification History
% Version	Modifier Date	Author			Change
% ====================================================================
% 0.0.1		2014-11-07		Tesser Paolo	inserita sezione
% ====================================================================
% 0.0.2		2015-02-03		Tesser Paolo	sistemata nota Amministratore, Ambiente di lavoro
%
%

\section{A}

\begin{itemize}
	\item \textbf{Ambiente di lavoro}: è ciò che serve ai processi di produzione. \'E composto da:
		\begin{itemize}
			\item persone
			\item ruoli
			\item procedure
			\item infrastrutture
		\end{itemize}
		\noindent
		Esso influisce sulla qualità del processo e del prodotto. Deve essere quindi: completo, ordinato e aggiornato;

	\item \textbf{Amministratore}: è uno dei ruoli di gestione di un progetto.
Esso ha il compito di controllare l'ambiente di lavoro e di amministrare le infrastrutture necessarie allo svolgimento del progetto. \newline
Deve mettere in pratica ciò che le norme chiedono e attraverso sistemi automatizzati le deve fare eseguire senza renderle troppo ingombranti per gli altri membri. Questo lavoro va fatto in maniera preventiva e pro attiva rispetto l'inizio delle attività e del tutto trasparente agli altri componenti. \newline
A questa figura fanno quindi capo le configurazioni sui sistemi di versionamento e tutta la documentazione che viene redatta. Non attua scelte personali, ma segue quelle concordate con i responsabili;
	\item \textbf{Analista}: è uno dei ruoli di gestione di un progetto.
Esso ha il compito di capire il problema per ottenere i requisiti. Non da la soluzione e spesso non segue fino in fondo la realizzazione del progetto, ma è presente principalmente nella fase iniziale. \newline
Per cercare i requisiti segue un approccio top-down. Parte infatti ad analizzare quelli espliciti del proponente fino a scinderli in vari sottogruppi gerarchici, trovandone nel frattempo anche di impliciti (il numero maggiore tra le due tipologie);


\end{itemize}
