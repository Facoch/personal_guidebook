% File: T.tex
% Created: 2014-11-07
% Author: Tesser Paolo
% Email: p.tesser921@gmail.com
%
%
% Modification History
% Version	Modifier Date	Author			Change
% ====================================================================
% 0.0.1		2014-11-07		Tesser-Paolo	inserita sezione
% ====================================================================
% 0.0.2		2015-03-18		Tesser Paolo	inserito vocabolo: Tecniche di analisi
% ====================================================================
%

\section{T} % (fold)
\label{sec:t}
	\begin{itemize}
		\item \textbf{Tecniche di analisi}: [IS] ci sono diversi approcci per analizzare i bisogni e le fonti di un progetto. Attraverso lo studio del dominio, andando quindi ad osservare i comportamenti dell'utente finale e dell'ambiente d'uso immergendosi nei panni degli utilizzatori. Interagendo con il cliente in maniera il meno invasiva possibile e facendolo in modo strutturato e rendicontabile. Un'altra tecnica è quella di avere dei brain-storming, ma per farlo bisogna che:
			\begin{itemize}
				\item ci sia un gruppo di persone che sono consapevoli di ciò che stanno facendo;
				\item serve un'altra persona che fa da controllore, fissa l'agenda, il tempo e la scaletta;
				\item serve una persona che tenga le minute, cioè i punti salienti.
			\end{itemize}
		\noindent
		Infine può essere fatta anche della prototipazione, che può essere interna (per capire meglio le cose) o esterna (per chiedere al cliente se ciò che sta domandando);

	\end{itemize}
% section t (end)