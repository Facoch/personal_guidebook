% File: R.tex
% Created: 2014-11-07
% Author: Tesser Paolo
% Email: p.tesser921@gmail.com
%
%
% Modification History
% Version	Modifier Date	Author			Change
% ====================================================================
% 0.0.1		2014-11-07		Tesser-Paolo	inserita sezione
% ====================================================================
% 0.0.2		2015-03-12		Tesser Paolo	inserita voce Requisito
% ====================================================================
%

\section{R}

\begin{itemize}
	\item \textbf{Requisito}: [IR] possono essere espliciti o impliciti. Secondo IEEE, è la condizione necessaria a un utente per risolvere un problema o raggiungere un obiettivo (visione dal lato del bisogno). \'E inoltre la condizione che deve essere soddisfatta o posseduta da un sistema per adempiere a un obbligo (visione dal lato della soluzione). \newline
	Il requisito è dunque necessario per: capire la domanda e capire la risposta;

	\item \textbf{Responsabile}: è uno dei ruoli di gestione di progetto. Esso rappresenta il progetto di tutto il team presso il fornitore e il committente. Viene quindi accentrata la responsabilità di scelta e approvazione. \newline
	Partecipa interamente al ciclo di vita del prodotto e sa sempre (senza bisogno di domandare nulla agli altri membri) a che punto è l'avanzamento delle attività per il raggiungimento degli obiettivi fissati. \newline
	Ha quindi responsabilità sulla pianificazione, sulla gestione delle risorse umane tramite coordinamento e relazioni esterne con il committente. \newline
 	pianificazione consiste nella definizione delle attività che il team dovrà svolgere e l'assegnamento dei tempi di esecuzione per quelle attività riuscendo a determinarne i costi di attuazione, da fornire poi come preventivo al proponente e come consuntivo nella fase finale. \newline
	Ci sono diversi strumenti per lo svolgimento di questo ruolo, come l'utilizzo di Diagrammi di Gantt e di PERT .

	\item \textbf{Ricetta generale} [PR] organizzare il sistema in componenti di complessità trattabile secondo la logica ``divide-et-impera''. Questo consente di diminuire la difficoltà di comprensione e di realizzazione, permettendo un maggiore lavoro individuale. Per facilitarsi in questo compito è utile individuare schemi architetturali e componenti riusabili. Andare poi a riconoscere le componenti terminali sotto la quale non bisogna più andare a scomporre. Infine cercare un buon bilanciamento in quanto più semplici e divise sono le componenti più complessa sarà la loro interazione e maggiore sarà l'accoppiamento;

	\item \textbf{Riuso}: [PR] utilizzare componenti già creati può permettere di capitalizzare sottoinsieme già esistenti. Impiegandoli più volte si ha un minor costo realizzativo e un minor costo di verifica. Questa progettazione però è difficile in quanto bisogna anticipare bisogni futuri. Nel breve periodo è quindi sicuramente un costo, ma farà risparmiare nel medio termine;

	\item \textbf{Ruolo}: è una funzione aziendale assegnata a progetto. E' un servizio che viene svolto per la comunità. Ne esistono di diversi e non tutti ammettono parallelismo. \newline
	Ci sono diversi ruoli che si possono raggruppare più macroscopicamente in quattro gruppi:
	\begin{enumerate}
		\item \textbf{Sviluppo}: ha la responsabilità tecnica e realizzativa;
		\item \textbf{Direzione}: ha la responsabilità decisionale (molto importante);
		\item \textbf{Amministrazione}: ha la responsabilità della gestione dei processi;
		\item \textbf{Qualità}: ha la responsabilità di gestione della qualità.
	\end{enumerate}

\end{itemize}
