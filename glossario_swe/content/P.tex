% File: P.tex
% Created: 2014-11-07
% Author: Tesser Paolo
% Email: p.tesser921@gmail.com
%
%
% Modification History
% Version	Modifier Date	Author			Change
% ====================================================================
% 0.0.1		2014-11-07		Tesser-Paolo	inserita sezione e lista contenente le voci del glossario
% ====================================================================
% 0.0.2		2015-02-24		Tesser Paolo	miglioramento vocabolo 'Processo'
% ====================================================================
%

\section{P}

\begin{itemize}
	\item \textbf{Piano di Progetto}: [PM] è uno dei documenti esterni da redarre. Viene steso dal responsabile di progetto. \newline
In esso vengono fissate le risorse disponibili, la suddivisione delle attività e il loro calendario. \newline
Si deve cercare di organizzare le attività in modo da produrre risultati utili per valutare con efficacia il grado di avanzamento del lavoro. \newline
Serve inoltre fissare delle milestone come punti critici o finali delle attività. \newline
La struttura tipica di questo documento segue questo schema:
	\begin{enumerate}
		\item Introduzione (scopo e struttura);
		\item Organizzazione del progetto;
		\item Analisi dei rischi, maggiori sono i rischi maggiori saranno i punti di controllo. \newline
Le tipologie di rischi possibili sono: a livello tecnologico, a livello del personale, a livello organizzativo, a livello dei requisiti (quello che causa più probabilità di fallimento) e a livello di valutazione dei costi;
		\item Risorse necessarie e risorse disponibili (HW e SW);
		\item Suddivisione del lavoro (Work Breakdown);
		\item Calendario delle attività;
		\item Meccanismi di controllo e di rendicontazione.
	\end{enumerate}
 

	\item \textbf{Pro attivi}: al contrario di re-attivi, si è pro attivi quando si cerca di risolvere un problema prima che esso si manifesti. In particolare riguarda gli errori che si potranno commettere e cercare di prevenirli. \newline
Agire prima permette di ridurre i costi oltre al fatto di capire meglio ciò che si sta facendo e garantire una migliore qualità del prodotto;

	\item \textbf{Procedura}: in generale è l'insieme di regole e norme da svolgere in maniera sequenziale per ottenere un determinato risultato. \newline
In informatica (funzione) è un blocco di istruzioni che a partire da un certo input restituiscono un determinato output;

	\item \textbf{Processo}:  in generale è una rete di cambiamenti, attività o azioni collegate tra loro per la creazione, la verifica, la manutenzione, ecc... di un prodotto. Esso riguarda il modo di lavorare. Un processo va dunque controllato per poterlo migliorare ed essere più efficace, quindi con prodotti conformi alle attese, e più efficiente, quindi con minori costi a pari qualità di prodotto. \newline
	Servono quindi buoni strumenti di valutazione che non siano invasivi e che forniscano valori in pull. \newline
	Durante il ciclo di vita del software tramite un processo si attua la transizione tra uno stato e l'altro. Nel particolare vengono chiamati processi di ciclo di vita. \newline
	In Informatica si intende un'istanza di un programma in esecuzione in modo sequenziale. A differenza di un programma, esso è un'entità dinamica, che dipende dai dati che vengono elaborati, e dalle operazioni eseguite su di essi. Il processo è quindi caratterizzato, oltre che dal codice eseguibile, dall'insieme di tutte le informazioni che ne definiscono lo stato, come il contenuto della memoria indirizzata, i thread, i descrittori dei file e delle periferiche in uso;
	
	\item \textbf{Progettista}: [PM] è uno dei ruoli di gestione di un progetto. \newline
	Esso ha il compito di trovare la migliore soluzione per i requisiti trovati dall'analista. Per fare ciò generalmente utilizza un approccio bottom-up. \newline
Non si occupa dell'implementazione, compito spettante invece al programmatore. La sua presenza nel progetto dura in genere fino alla manutenzione.
	
	\item \textbf{Programmatore}: [PM] è uno dei ruoli di gestione di un progetto. 	\newline
Esso ha il compito di implementare le soluzioni trovate dal progettista. Non si deve inventare nulla proprio perché quello che deve fare è già stato fissato in precedenza. Partecipano sia allo sviluppo che alla manutenzione.

	\item \textbf{Programma}: è un insieme di istruzioni che, una volta eseguite su un computer, produce soluzioni per una data classe di problemi automatizzati;

\end{itemize}